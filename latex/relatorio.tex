\documentclass{article}

\usepackage[utf8]{inputenc}
\usepackage[portuguese]{babel}
\usepackage[T1]{fontenc}
\usepackage{lmodern}
\usepackage{fullpage}
\usepackage[usenames]{color}
\usepackage[color]{coqdoc}
\usepackage{url}
\usepackage{makeidx,hyperref}
\usepackage[all]{xy}

\usepackage{mathpartir}
\usepackage{tcolorbox}
\usepackage{amsmath,amssymb}

\newcommand{\tto}{\twoheadrightarrow}
\newcommand{\ott}{\twoheadleftarrow}

\title{Formalização da correção do algoritmo Bubblesort}
\author{Flávio}
\date{\today}

\begin{document}
\maketitle

\section{Introdução}

Este trabalho apresenta uma prova formal da correção do algoritmo de ordenação por borbulhamento ({\em bubblesort}). A formalização foi feita no assistente de provas Coq.

O assistente de provas Coq utiliza o sistema de Dedução Natural, o que o torna adequado para o desenvolvimento de atividades computacionais no curso de Lógica Computacional 1.

O Coq permite a extração de código certificado em diversas linguagens funcionais, como Ocaml, Haskell e Scheme. Ao final do trabalho exploraremos a ferramenta de extração de código.

\input{bubble_sort.v}

\section{Conclusão}

Coloque aqui a conclusão do relatório

\end{document}
